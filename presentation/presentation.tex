% !TEX TS-program = pdflatex
% !TEX encoding = UTF-8 Unicode

\documentclass{beamer}


\mode<presentation>
{
  \usetheme{default}
  % or ...

  \setbeamercovered{transparent}
  % or whatever (possibly just delete it)
}


\usepackage[english]{babel}
\usepackage[utf8]{inputenc}

\usepackage{times}
\usepackage[T1]{fontenc}
\usepackage{fancyvrb}
% Or whatever. Note that the encoding and the font should match. If T1
% does not look nice, try deleting the line with the fontenc.

% \title[Short Paper Title] % (optional, use only with long paper titles)

\title{Ants for Dinner}
\subtitle{Programming an ants strategy}

\author{Laurens van den Brink, Philipp Hausmann, Marco Vassena}
% - Give the names in the same order as the appear in the paper.
% - Use the \inst{?} command only if the authors have different
%   affiliation.

\date{30 October 2013}
% - Either use conference name or its abbreviation.
% - Not really informative to the audience, more for people (including
%   yourself) who are reading the slides online


% Delete this, if you do not want the table of contents to pop up at
% the beginning of each subsection:
\AtBeginSubsection[]
{
  \begin{frame}<beamer>{Outline}
    \tableofcontents[currentsection]
  \end{frame}
}

% If you wish to uncover everything in a step-wise fashion, uncomment
% the following command: 

%\beamerdefaultoverlayspecification{<+->}


\begin{document}

\begin{frame}
  \titlepage
\end{frame}

%\begin{frame}{Outline}
%  \tableofcontents
%  % You might wish to add the option [pausesections]
%\end{frame}

\section{Programming an ants strategy}
\begin{frame}{Programming an ants strategy approach}
	\begin{itemize}
		\item Custom syntax
		\item Parsing
		\item Abstract syntax tree
		\item Compile
		\item Output ant strategy
	\end{itemize}
\end{frame}

\begin{frame}[fragile]{Syntax}
	\begin{block}{Example}
		% SCREENSHOT
	\end{block}
\end{frame}

\begin{frame}{Compiler}
	\begin{itemize}
		\item Compile Syntax tree
		\item Function calls
		\item Recursive functions
	\end{itemize}
\end{frame}

\section{Demo}

\begin{frame}{Demo}
	\begin{itemize}
		\item The strategy
		\item Demonstration
	\end{itemize}
\end{frame}


\end{document}


