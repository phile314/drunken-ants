% !TEX TS-program = pdflatex
% !TEX encoding = UTF-8 Unicode

\documentclass{beamer}


\mode<presentation>
{
  \usetheme{default}
  % or ...

  \setbeamercovered{transparent}
  % or whatever (possibly just delete it)
}


\usepackage[english]{babel}
\usepackage[utf8]{inputenc}

\usepackage{listings}
\usepackage{times}
\usepackage[T1]{fontenc}
\usepackage{fancyvrb}
% Or whatever. Note that the encoding and the font should match. If T1
% does not look nice, try deleting the line with the fontenc.

% \title[Short Paper Title] % (optional, use only with long paper titles)

\title{Ants for Dinner}
\subtitle{Programming an ants strategy}

\author{Laurens van den Brink, Philipp Hausmann, Marco Vassena}
% - Give the names in the same order as the appear in the paper.
% - Use the \inst{?} command only if the authors have different
%   affiliation.

\date{30 October 2013}
% - Either use conference name or its abbreviation.
% - Not really informative to the audience, more for people (including
%   yourself) who are reading the slides online



% Delete this, if you do not want the table of contents to pop up at
% the beginning of each subsection:
\AtBeginSubsection[]
{
  \begin{frame}<beamer>{Outline}
    \tableofcontents[currentsection]
  \end{frame}
}

% If you wish to uncover everything in a step-wise fashion, uncomment
% the following command: 

%\beamerdefaultoverlayspecification{<+->}


\begin{document}

\begin{frame}
  \titlepage
\end{frame}

%\begin{frame}{Outline}
%  \tableofcontents
%  % You might wish to add the option [pausesections]
%\end{frame}

\section{Programming an ants strategy}
\begin{frame}{Approach}
	Design and implement a DSL, which allows to define a strategy in a natural way.
\end{frame}

\begin{frame}[fragile]{DSL}
	Our custom language provides imperative features:
	\begin{itemize}
		\item Blocks and statements
		\item For Loop
		\item If - Then - Else
		\item Scoped bindings (variables and local functions)
		\item Try - Catch
		\item Top level declarations
		\item Procedures
		\item Mutual tail recursion
		\item Modules
	\end{itemize}
\end{frame}

\lstset{basicstyle=\footnotesize}
\begin{frame}
	\begin{block}{Example}
        \lstinputlisting{../example-ants/correct/pres_ant.ha}
	\end{block}
\end{frame}


\begin{frame}{Parser}

	\begin{itemize}
		\item Matches the input file with the grammar of the language
		\item Constructs the abstract syntax tree (AST).
		\item Loads and parses recursively any imported module.
		\item Uses Parsec library.
	\end{itemize}
	
\end{frame}

\begin{frame}{Compiler}
	\begin{itemize}
		\item Compiles the syntax tree into the assembly code
		\item Inline bindings
		\item Handles function calls and recursion
		\item Reports errors
	\end{itemize}
\end{frame}

\section{Strategy}

\begin{frame}{Strategy}
	Essential strategy:
	\begin{enumerate}	
		\item Random walk
		\item Pick up food
		\item Go back home
		\item Drop food
	\end{enumerate}
\end{frame}

\begin{frame}{Further Work}
    \begin{itemize}
        \item Duplicated code elimination
        \item More syntactic sugar (while, else-if, switch statement)
        \item Relax recursion constraints (allow parameters)
       	\item Extend variables
    \end{itemize}
\end{frame}

\end{document}
